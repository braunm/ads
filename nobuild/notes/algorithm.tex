\documentclass[letter,10pt]{article}

\usepackage[pdftex]{graphicx}

\usepackage{geometry}
\usepackage{natbib}
\usepackage[singlespacing]{setspace}
\usepackage[page]{appendix}
\usepackage[color]{showkeys}

\definecolor{labelkey}{rgb}{1,.5,0.2}
\definecolor{refkey}{rgb}{0,0,0.8}

\usepackage[T1]{fontenc}
%\usepackage{float}
\usepackage{amsmath}
%%\usepackage{amsfonts}
\usepackage{amssymb}
\usepackage{verbatim}
\usepackage{mathpazo}
%%\usepackage[charter]{mathdesign}
\usepackage{caption}

\bibpunct[, ]{(}{)}{;}{a}{}{,}
\bibliographystyle{ormsv080}

\DeclareMathOperator\logit{logit}
\DeclareMathOperator\Prob{Prob}
\DeclareMathOperator\vecop{vec }
\DeclareMathOperator\tr{tr}
\DeclareMathOperator\etr{etr}


\newcommand{\thetabar}{\bar\theta}
\newcommand{\thetaij}{\theta_{ij}}
\newcommand{\kron}{\otimes}
\newcommand{\eps}{\varepsilon}
\newcommand{\Igtz}{\mathbb{I}}
%%\newcommand{\Igtz}{\mathbf{I}}
\newcommand{\qbar}{\bar{q}}


\newcommand{\relphantom}[1]{\mathrel{\phantom{#1}}}

\geometry{left=1in,right=1in,top=1in,bottom=1in}

\title{Implementation notes for HDLM ad spend model}
\author{Andr\'e Bonfrer and Michael Braun}
\date{April 15, 2015}

\begin{document}
\maketitle


\section*{Definitions of symbols}
\begin{table}[h]\centering
\begin{tabular}{cp{4in}l}
Symbol&Definition&Dimension\\
\hline
$l = 1\mathellipsis L $& hierarchy level index (level 1 is city, level 2 is national)  &\\
$j=1\mathellipsis J$&brand index&\\
$i=1\mathellipsis N$&city index&\\
$t=1\mathellipsis T$&time index&\\
$k=1\mathellipsis K$&time invariant covariate index at level 1&\\
$p=1\mathellipsis P$&time varying covariate index at level 1&\\
\hline
&{\bf Observed data}&\\
$Y_t$&dependent variable.  Sales by city and brand&$N\times J$\\
$F_{11t}$&Data ('explanatory variables') matrix.  Sparse 0/1&$N\times N(1+P)$\\
$F_{12t}$&Data ('explanatory variables') matrix.  Sparse 0/1&$N\times K$\\
$F_{2t}$&Hierarchical matrix.  &$N(1+P)\times (1+J+P)$\\
$A_t$&ad spend by city, brand&$N\times J$\\
$F_{12t}$&covariate matrix for the 1st hierarchy level (by city, time varying, but observed)&$N\times P$\\
\hline
&{\bf Prior parameters (predetermined)}&\\
$M_{20}$&some prior matrix-normal parameter for
$\Theta_{20}$&$(1+J+P)\times J$\\
$C_{20}$&some prior matrix-normal parameter for $\Theta_{20}$&$(1+J+P)\times(1+J+P)$\\
$\Omega_0$&prior inverse Wishart scale parameter matrix for $\Sigma$& $J\times J$\\
$\nu_0$&prior inverse Wishart degrees of freedom parameter for $\Sigma$&scalar\\
&other hyperprior parameters go here&\\
\hline
&{\bf parameters that need to be estimated}&\\
$\Theta_{12}$&time invariant coefficients on city and brand specific covariates, non time varying (covariates are $F_{12t})$&$K\times J$\\
$\Theta_{11t}$&time varying coefficients on city and brand specific covariates (covariates are $F_{11t})$&$N(1+P)\times J$\\
$\Theta_{2t}$&time varying (mean) coefficients (covariates can be introduced in $F_{2t})$&$(1+J+P)\times J$\\
$c$&vector of copy wearout parameters, brand-specific&$J$\\
$u$&vector of ad wearout parameters, brand-specific&$J$\\
$\phi$&matrix of ad wearout parameters, brand-specific&$J\times J$\\
$\delta$&decay parameter, $0<\delta <1$&scalar\\
$V_1$&city level covariance matrix&$N\times N$\\
$V_2$&hierarchical covariance matrix (across model components)&$N(1+P)\times N(1+P)$\\
$W$&evolution covariance matrix (over time)&$(1+J+P)\times (1+J+P)$\\
\hline
&{\bf Intermediate terms}&\\
$\tilde{G}_t$&evolution matrix (upper triangular)&$(1+J+P)\times(1+J+P)$\\
$\tilde{H}_t$&matrix depending on $\delta$ and $A_t$&$(1+J+P)\times J$\\
$a_{2t}$&mean for prior distribution of $\Theta_{2t} \mid D_{t-1}$&$(1+J+P)\times J$\\
%$a_{1t}$&&$N(J+1)\times J$\\
$f_t$&mean of predictive distribution for $Y_t\mid D_{t-1}$ &$N\times J$\\
$Q_t$&covariance of predictive distribution for $Y_t\mid D_{t-1}$&$N\times N$\\
%$M_{1t}$&not needed? Can we remove?&$N(J+1)\times J$\\
$R_{1t}$&covariance for prior distribution of $\Theta_{1t}\mid D_{t-1}$&$N(1+P)\times N(1+P)$\\
$R_{2t}$&covariance for prior distribution of $\Theta_{2t}\mid D_{t-1}$&$(1+J+P)\times(1+J+P)$\\
$M_{2t}$&mean of posterior distribution $\Theta_{2t}\mid D_{t}$ &$(1+J+P)\times J$\\
$C_{2t}$&covariance for posterior distribution $\Theta_{2t}\mid D_{t}$ &$(1+J+P)\times (1+J+P)$\\
%$S_{1t}$&not needed? Can we remove?&$N(J+1)\times N$\\
$S_{2t}$&used to translate variance from previous level &$(1+J+P)\times N$\\
%$\Omega_t$&prior inverse Wishart scale parameter matrix for $\Sigma\mid D_{t-1}$& $J\times J$\\
%$\nu_t$&prior inverse Wishart degrees of freedom parameter for $\Sigma\mid D_{t-1}$&scalar\\
\end{tabular}
\end{table}

\section{Two-level hierarchical model}

Assume we have outcome (e.g sales) data for $J$ brands, in $N$ cities, and observe these data
over $T$ time periods.   We represent this outcome ($Y_t$) at time $t$ as a $N\times J$ matrix, 
with the rows representing city level 
observations, and columns representing brand sales (outcome) data.  
We have a hierachy of city level at the lowest (each city has its own sales) and at the
highest level we have dynamics at the mean level (e.g. mean prices, or the effects of national level advertising).  
This is written as:
\begin{eqnarray}
Y_{t} & = & F_{11t} \Theta_{11t} + F_{12t} \Theta_{12} + v_{1t} \\
\Theta_{11t} & = & F_{2t}\Theta_{2t} + v_{2t}  \\
\Theta_{2t} & = & \tilde{G}_t \Theta_{2,t-1} + \tilde{H}_t + w_t 
\end{eqnarray}

The components that affect each city's sales directly, are in the $F_{12t}$ matrix, with a corresponding 
non-time varying coefficient matrix.   The time varying component at the city level is contained in the 
$F_{11t} \Theta_{11t}$ component.  In addition we have an innovation function (sometimes called a control variable) in the 
evolution equation, $H_t$.  This component shifts elements of $\Theta_{2t}$ but is not relative to it.  

We use a matrix normal distribution for all covariance terms:
\begin{eqnarray} 
v_{lt} \mid \Sigma, V_l \sim N(0,V_l,\Sigma)\\
w_{t} \mid \Sigma, W \sim N(0,W,\Sigma)
\end{eqnarray}
Each matrix normal distribution has a left and right variance matrix, e.g. $V_1,\Sigma$ respectively.  
The right variance governs (column) cross equation covariatance.  The left variance captures row covariance, which is either
concurrent ($V_l$) or time based variation ($W$).  The left variance  $V_1$ represents variation across
cities.  The left variance $V_2$ represents concurrent variance across mean values for different state variables.  
We can simplify notation a bit by using a set $\Psi = \{V_1,V_2,W\}$.

At the first level we have $\bar{Y}_t = Y_t - F_{12t} \Theta_{12}$, which does not 
have a hierarchical counterpart (i.e. homogenous response to covariates contained in $F_{12t}$).  
The $\Theta_{11t}$ component then has both time varying and non-time varying 
heterogeneous responses.  
We can rewrite our HDLM as:\begin{eqnarray}
\bar{Y}_{t} & = & F_{11t} \Theta_{11t} + v_{1t} \\
\Theta_{11t} & = & F_{2t}\Theta_{2t} + v_{2t}  \\
\Theta_{2t} & = & \tilde{G}_t \Theta_{2,t-1} + \tilde{H}_t + w_t 
\end{eqnarray}
We use the tilde ($\tilde{\cdot}$) in the above to represent intermediate variables (those that depend on other parameters).  

\section*{Data likelihood}

We write the data likelihood, recognising that 
simple substitutions can be made to take care of the time invariant and homogenous parameters.
Let $D_{t-1}$ represent all information (including state variables) available at time $t$.  So $D_0$ is initial 
information about the states and priors.  
At any time period, $\Sigma \mid D_{t-1}$ is distributed as an Inverse Wishart ($IW(\nu_{t-1},\Omega_{t-1}$).
For any time period, the joint density of the data $Y_t$ and $\Sigma$ is a matrix
normal inverse Wishart (or a product of a matrix normal with inverse Wishart):
\begin{align}
P(\bar{Y},\Sigma\mid \mathcal{D}_0,\Psi)&=\prod_{t=1}^T P(\bar{Y}_t \mid \Sigma, D_{t-1},\Psi) P(\Sigma \mid D_{t-1},\Psi) \\
%&=  \prod_{t=1}^T   (2\pi)^{-\frac{NJ}{2}}|Q_t|^{-\frac{J}{2}}|\Sigma|^{-\frac{N}{2}}\exp\left[
%  -\frac{1}{2}\tr\left(\left(
%      Y_t-f_t\right)'Q_t^{-1}\left(Y_t-f_t\right)\Sigma^{-1}\right)\right]\\
& = \prod_{t=1}^T   (2\pi)^{-\frac{NJ}{2}}|Q_t|^{-\frac{J}{2}}|\Sigma|^{-\frac{N}{2}}\exp\left[
  -\frac{1}{2}\tr\left(\left(
      \bar{Y}_t-f_t\right)'Q_t^{-1}\left(\bar{Y}_t-f_t\right)\Sigma^{-1}\right)\right] \nonumber\\
& ~~~~~~ \times IW(\nu_{t-1},\Omega_{t-1})
\end{align}
Integrating out $\Sigma$ (see our technical appendix on matrix T) gives the following data likelihood:
\begin{align}
  P(\bar{Y}|\cdot)&=\prod_{t=1}^TP(\bar{Y}_t|y_{1:t-1},\cdot)\nonumber\\
\label{eq:LL-T}
&= \mathcal{K} \left(\prod_{t=1}^T|Q_t|^{-\frac{J}{2}}\right)|\Omega_0+\sum_{t=1}^T\left(\bar{Y}_t-f_t\right)'Q_t^{-1}\left(\bar{Y}_t-f_t\right)|^{-\frac{\nu_0+TN}{2}}
\end{align}
where:
\[
\mathcal{K} = \pi^{-\frac{NJT}{2}}\frac{\Gamma_J\left(\frac{\nu_0+TN}{2}\right)}{\Gamma_J\left(\frac{\nu_0}{2}\right)}|\Omega_0|^{-\frac{\nu_0}{2}}
\]




\section{Theoretical development of the dynamic hierarchical effects of advertising}

In our application (FMCG), we take the perspective of a national-level marketing manager observing city  
level observations of sales and prices, allocating an advertising expenditure over time and across cities.  
We focus on the effectiveness of network advertising, which as a unit of analysis is observed only at a national level.
That is, each city is exposed to the same network television patterns.  Effectiveness is a dynamic function of 
the content in the creatives used in the campaign.  At any time, there are a number of distinct creatives
observed in the market place.  The manager can choose to spend on existing/past creatives, or to invest in 
and launch a new creative.  The industry setting is such that there 
are a total of $J$ competitors.  Expenditure allocation decisions are made on a continuous time basis but 
we use weekly level sales data so we convert the expenditures to a weekly level.  

In the full competitive model, all covariates (advertising, price and promotions) have both an own
and cross effect.   In the matrix normal set up of
the HDLM above, this is automatically specified by having a matrix normal of the brand sales in the columns,
and the covariates are each brands' covariates.  

In addition to competitive effects and competitive interference effects, the effectiveness of
the advertising campaign wears out over time.  This is because of the wearout of individual messages.  
In line with past work (e.g. Naik et al 1998,  Bass et al 2008, Braun and Moe 2012), we think 
of ad effectiveness for each brand $j$ creative $m$, denoted by $q_{mjt}$.  

We abstract from any individual message and 
start with the standard Nerlove-Arrow type evolution of "brand" or overall "advertising"
effectiveness (defined as the ability for a dollar spent on advertising to lift sales volume), represented by:
\begin{equation}
\label{eqn:bqj1}
B_{jt} = (1-\delta) B_{jt-1} + q_{jt-1} g(A_{ijt})  
\end{equation}
where $g(\cdot)$ is some transformation of advertising ($=0$ if
$A_{ijt}=0$).
We will return to the issue of including competitive effects of a focal brand's advertising, multiple media messages
and the effect of competing brands' advertising on the focal brand's sales. 

One way to build dynamic advertising effectiveness is by using a differential equation with respect
to time, of brand $j$'s advertising:
\begin{align}
\frac{dq_j}{dt} = -a(A_j)q_j +(1-\Igtz(A_j)) \delta (\qbar_j - q_j) 
\end{align}

If $a(A_j)=c_j+u_jA_j$, then

\begin{align}
  \label{eq:1}
  \frac{dq_j}{dt} = -\left(c_j+u_jA_j\right)q_j +(1-\Igtz(A_j)) \delta (\qbar_j - q_j) 
\end{align}


Solving this differential equation gives us
\begin{align}
  \label{eq:2}
  q_j(t)&=
\begin{cases}
\dfrac{\delta\qbar}{c_j+\delta}+Ke^{-\left(c+\delta\right)t}&A_j=0\\
Ke^{-\left(c+u_jA_j\right)t}&A_j>0
\end{cases}
\end{align}

If there is no ad spend,
$\lim\limits_{t\to\infty}q_j(t)=\dfrac{\delta\qbar}{c_j+\delta}$. When
there is ad spend, $\lim\limits_{t\to\infty}q_j(t)=0$.

In discrete time, 
\begin{align}
  \label{eq:3}
  q_{jt}-q_{j,t-1}&=-\left(c_j+u_jA_j\right)+\delta\left(1-\Igtz(A_j)\right)\left(\qbar-q_{j,t-1}\right)\\
&=-\left[c_j+u_jA_j+\delta\left(1-\Igtz(A_j)\right)\right]q_{j,t-1}+\delta\left(1-\Igtz(A_j)\right)\qbar\\
q_{jt}&=\left[1-c_j-u_jA_j-\delta\left(1-\Igtz(A_j)\right)\right]q_{j,t-1}+\delta\left(1-\Igtz(A_j)\right)\qbar
\end{align}


START HERE



A few notes about this.  First, it is only valid if $q>0$. For the cross effectiveness of 
advertising in our competitive model, we need to allow for this to be negative.  If $q<0$ then we need a different
dynamic:
\[
\frac{dq_j}{dt} = -a(A_j) - (1-\Igtz(A_j)) \delta (1 + q_j) 
\]
which allows advertising effectiveness to return to a larger negative value if advertising for
that brand is switched off.  If advertising is switched off, it has the same asymptote as before, 
but negative in value\footnote{\baselineskip 12pt Our issues with the identification of $c$ 
means we set this to $0$.  The obvious implication is that the asymptotic value of $q$ is then 
equal to one.}:
\[
\lim_{t \rightarrow \infty} q_{jt} ~=  -\frac{\delta}{c+\delta}
\]

This leads to the following:
\begin{eqnarray}
q_{jt} - q_{jt-1} & = & -a(A_j) q_{jt-1} - (1-\Igtz(A_j))\delta (1+ q_{jt-1})\nonumber\\
q_{jt} & = & q_{jt-1} -a(A_j) q_{jt-1} - (1-\Igtz(A_j))\delta (1+ q_{jt-1})\nonumber\\
& = & q_{jt-1} \left[1-a(A_j) -\delta (1-\Igtz(A_j))\right]  - \delta (1- \Igtz(A_j))
\label{eqn:dq2}
\end{eqnarray}
The only difference between (\ref{eqn:dq1}) and (\ref{eqn:dq2}) is the "innovation" component ( $sign(q) \delta(1-\Igtz(A_j))$)
which depends on the sign of the advertising effectiveness component.  See below for how
we include the innovation component (it does not enter $\tilde{G}_t$).

Second, we want to incorporate new creatives.  For this, we propose the differential equation:
\[
\frac{dq_j}{dt} = -a(A_j)q_j +(1-\Igtz(A_j)) \delta (1 - sign(q_j) q_j) + sign(q_j) \phi_j E_j
\]
where $E_j$ is some metric for new creatives.  E.g. it could count the number of new
creatives being used (or the proportion of the creatives that have not been shown before).  


\subsection{Specifying $\Theta_{2t}$}

The matrix $\Theta_{2t}$ is a (dense)  matrix of time varying parameters 
(also called state variables).  Without the $P\times J$ matrix of time varying parameters, the rows of this correspond to 

\begin{equation}
\begin{array}{ll}
      \Theta_{2t} = & \left[\begin{array}{cccl}
		 B_{1t} & B_{2t} &\ldots & B_{Jt} \\
		q_{11t} & q_{21t} & \ldots & q_{J1t} \\
		q_{12t} & q_{22t} & \ldots & q_{J2t} \\
	\vdots\\
		q_{1Jt} & q_{21t} & \ldots & q_{JJt} \\

		\end{array}\right]
\end{array}
\label{eqn:t2a}
\end{equation} 
with $F_{2t}$ being the matrix (dimension $N \times (J+1)$) that translates these states to the city level.  For example, corresponding
to the above:
\begin{equation}
\begin{array}{ll}
      F_{2t} = & \left[\begin{array}{cccl}
		1 & 0 & 0 & 0\\
		1 & 0 & 0 & 0\\
\vdots\\
		1 & 0 & 0 & 0\\
		\end{array}\right]
\end{array}
\label{eqn:t2b}
\end{equation} 
The zero elements of this matrix make the ad effectiveness parameters/states latent, with their
role only being played in the evolution matrix below.   For (\ref{eqn:t2a}) and (\ref{eqn:t2b}) they combine
to provide a $N\times J$ matrix which just depends on the intercept ($B_{jt}$ in \ref{eqn:bqj1}).  

\subsubsection{Adding time varying effects of other covariates}

To add time varying effects of other covariates, we add in rows corresponding to each effect to $\Theta_{2t}$, e.g. 
adding $J$ rows\footnote{\baselineskip 12pt This is a bit confusing, but because we have cross effects for each covariate we add, then 
adding in just price adds $P=J$ covariates. If we allow two types of covariates (e.g. price and promotion) then
$P=J+J$}, 
a price "mean" parameter for each price variable (i.e. this is the mean value for the price elasticity across cities): 
\begin{equation}
\begin{array}{ll}
      \Theta_{2t} = & \left[\begin{array}{cccl}
		 B_{1t} & B_{2t} &\ldots & B_{Jt} \\
		q_{11t} & q_{21t} & \ldots & q_{J1t} \\
		q_{12t} & q_{22t} & \ldots & q_{J2t} \\
	\vdots\\
		q_{1Jt} & q_{21t} & \ldots & q_{JJt} \\

		\theta^p_{11t} & \theta^p_{21t} & \ldots & \theta^p_{J1t} \\
		\theta^p_{12t} & \theta^p_{22t} & \ldots & \theta^p_{J2t} \\
	\vdots\\
		\theta^p_{1Jt} & \theta^p_{21t} & \ldots & \theta^p_{JJt} \\

		\end{array}\right]
\end{array}
\end{equation} 
Then $F_{2t}$ becomes (with dimension $N(1+P)\times (1+J+P)$) :
\begin{equation}
\begin{array}{ll}
      F_{2t} = & \left[\begin{array}{cccccccc}
		1 & 0 & 0 & 0 & 0 & 0 & 0 & 0 \\
		0 &0 &0 &0 &1 &\ldots &0 &0\\
		0 & 0 & 0 & 0 & 0 &1 & 0 &0\\
		0 & 0 & 0 & 0 & 0 & 0 &\ddots &0\\
		0 & 0 & 0 & 0 & 0 &0 &0 & 1\\
\vdots\\
		1 & 0 & 0 & 0 & 0 & 0 & 0 & 0 \\
		0 &0 &0 &0 &1 &\ldots &0 &0\\
		0 & 0 & 0 & 0 & 0 &1 & 0 &0\\
		0 & 0 & 0 & 0 & 0 & 0 &\ddots &0\\
		0 & 0 & 0 & 0 & 0 &0 &0 & 1\\
\vdots\\
		1 & 0 & 0 & 0 & 0 & 0 & 0 & 0 \\
		0 &0 &0 &0 &1 &\ldots &0 &0\\
		0 & 0 & 0 & 0 & 0 &1 & 0 &0\\
		0 & 0 & 0 & 0 & 0 & 0 &\ddots &0\\
		0 & 0 & 0 & 0 & 0 &0 &0 & 1\\
		\end{array}\right]
\end{array}
\end{equation} 


\subsection{Specifying $F_{11t}$}

The city level matrix of $F_{11t}$ is the mean for the distribution of $\bar{Y}_t$ as a (simple additive) function
of the covariates (and latent space) at the city level.  Without any additional time varying effects of covariates,
this is an $N\times N$ identity matrix:
\begin{equation}
\begin{array}{ll}
      F_{11t} = & \left[\begin{array}{cccl}
		 1 & 0 & \ldots & 0 \\
		0 & 1 & \ldots & 0 \\
\vdots & 0 & \ddots \\
		0 & 0 & \ldots & 1 \\

		\end{array}\right]
\end{array}
\end{equation}

The addition of city specific price covariates, for which the effects vary over time, then the matrix is $N\times N(1+P)$
where $P$ corresponds to the number of covariates (times $J$).  For example, consider adding price for $J$ brands,
we will add $P=J$ covariates to $F_{11t}$:
\begin{equation}
\begin{array}{ll}
      F_{11t} = & \left[\begin{array}{cccccccccccc}
		 1 & p_{11t} & p_{1jt} & \ldots &p_{1Jt}\ldots & 0 &&&&&\ldots &0\\
\vdots\\
		0 &0 &0 &0 &\ldots & 1 & p_{n1t} & p_{njt} & \ldots &p_{nJt}  & \ldots & 0 \\
\vdots & 0 & \ddots \\
		\end{array}\right]
\end{array}
\end{equation}
where $p_{njt}$ is price in city $n$ for brand $j$ at time $t$.   In the above, the dimension would be $N \times N(1+P)$.


\subsection{Specifying $\tilde{G}_t$}

The corresponding "evolution" matrix $\tilde{G}_t$ is $(J+1+P)\times(J+1+P)$, and is upper triangular.  We illustrate
this below by ignoring the $\tilde{G}_t$ component for the $P$ time varying components (which would just be an identity matrix 
of dimension $P \times P$). Let
$\tilde{g}(A_{jt})$ be some transformation function of ad spend for
brand $j$ at time $t$.  The function $\Igtz(\cdot)$ is an indicator
function that takes a value of 1 if the argument is greater than zero,
and 0 otherwise.\footnote{If we allow for different "cross" effect wearouts, then we have an evolution matrix for each brand.  We would no longer
be able to use the matrix normal T distribution for this, and would have the following $j$th evolution matrix 
premultiplying the $j$th column of $\Theta_{2t}$:
\begin{equation}
\begin{array}{ll}
      \tilde{G}_{jt} = & \left[\begin{array}{cccl}
		 (1-\delta) & \tilde{g}(A_{1t}) & \ldots & \tilde{g}(A_{Jt}) \\
		0 & (1-c_{1j}-u_{1j} A_{1t}) - \delta(1-\Igtz(A_{1t})) & \ldots & 0 \\
\vdots & 0 & \ddots \\
		0 & 0 &  & (1-c_{Jj}-u_{Jj} A_{Jt} - \delta(1-\Igtz(A_{Jt})) \\
		\end{array}\right]
\end{array}
\end{equation}}  

\begin{equation}
Here\begin{array}{ll}
      \tilde{G}_t = & \left[\begin{array}{cccl}
		 (1-\delta) & \tilde{g}(A_{1t}) & \ldots & \tilde{g}(A_{Jt}) \\
		0 & (1-c_1 - u_1 A_{1t}) - \delta(1-\Igtz(A_{1t})) & \ldots & 0 \\
\vdots & 0 & \ddots \\
		0 & 0 &  & (1-c_J - u_J A_{Jt}) - \delta(1-\Igtz(A_{Jt})) \\
		\end{array}\right]
\end{array}
\end{equation}




\subsection*{Computing $H_t$}

The innovation component adds an amount to each state parameter, and for each 
brand.  Therefore the dimension of the matrix $\tilde{H}_t$ is $(1+J+P)\times J$, and depends 
on ad spend and the parameter $\delta$.  Again ignoring any additional time varying effects of 
covariates (so $P=0$) we have:
\begin{equation}
H^1_t = \left[ \begin{array}{ccccc}
	0 & \ldots & 0 & \ldots &0\\
	sign(q_{11}) \delta(1-\Igtz(A_{1t}))   &\ldots &sign(q_{j1}) \delta(1-\Igtz(A_{1t})) &\ldots &sign(q_{J1}) \delta(1-\Igtz(A_{1t}))  \\
\vdots &\ddots \\
	sign(q_{1j}) \delta(1-\Igtz(A_{jt}))     &sign(q_{kj}) \delta(1-\Igtz(A_{jt}))\\
\vdots &&& \ddots\\
	sign(q_{1J}) \delta(1-\Igtz(A_{Jt})) & \ldots &&&	sign(q_{JJ}) \delta(1-\Igtz(A_{Jt}))\\
	\end{array}
\right]
\end{equation}

\begin{equation}
H^2_t = \left[ \begin{array}{ccccc}
	0 & \ldots & 0 & \ldots &0\\
	sign{q_{11}}\phi_{11} E_{1t} & sign{q_{21}}\phi_{21} E_{1t} &\ldots &&sign{q_{J1}}\phi_{J1} E_{1t}\\
\vdots &\ddots \\
	sign{q_{1j}}\phi_{1j}  E_{jt}& \ldots &sign{q_{jj}}\phi_{jj} E_{jt} &\ldots &sign{q_{Jj}}\phi_{Jj} E_{jt}\\
\vdots\\
	sign{q_{1J}}\phi_{1J} E_{Jt}& sign{q_{2J}}\phi_{2J} E_{Jt} &\ldots &&sign{q_{JJ}}\phi_{JJ} E_{Jt}\\
	\end{array}
\right]
\end{equation}
Then $\tilde{H}_t = H^1_t + H^2_t$.  Once again, there will be an additional $P$ rows to $\tilde{H}_t$ corresponding
to additional time varying covariates.  



\section*{Iterative estimation}

To estimate the likelihood in Equation (\ref{eq:LL-T}), we need to
compute $\bar{Y}_t$, $f_t$, and $Q_t$.  The matrix $Y_t$ is the observed,
dependent variable, so we need to get $f_t$ and $Q_t$, and $\bar{Y}_t = Y_t - F_{12t} \Theta_{12}$.  
Conditional on estimates from time $t-1$, and all data and prior information, we can
follow the following algorithm at time $t$.  

\begin{enumerate}
\item Compute $\tilde{G}_t$ using $A_t$, $c, u$, $\phi$ and $\delta$, then $\tilde{G}_t$ to include any 
additional time varying effects for covariates. 
Similarly, create $\tilde{H}_t$ including $P\times J$ matrix of zeros.  
\item Set $a_{2t}=\tilde{G}_tM_{2,t-1}+\tilde{H}_t$.
\item Set $f_t=F_{11t}F_{2t}a_{2t}$.
\item Set $R_{2t}=\tilde{G}_tC_{2,t-1}\tilde{G}_t'+W$
\item Set $R_{1t}=F_{2t}R_{2t}F_{2t}'+V_2$
\item Set $Q_t=F_{11t}R_{1t}F_{11t}'+V_1$
\item Set $S_{2t}=R_{2t}\left[F_{11t}F_{2t}\right]'$
\item Set $M_{2t}=a_{2t}+S_{2t}Q_t^{-1}(\bar{Y}_t-f_t)$
\item Set $C_{2t}=R_{2t}-S_{2t}Q_t^{-1}S_{2t}'$
%\item Set $\nu_t = \nu_{t-1} + N$
%\item Set $\Omega_t = \Omega_{t-1} + E_t' Q_t^{-1} E_t$
\end{enumerate}
Then, iterate over $t$ to estimate the data likelihood.  Note that the homogenous time invariant component 
at level 1 of the hierarchy is handled by the transformed variable, $\bar{Y}_t$ which appears in the posterior and can 
be numerically estimated.    

\section*{Data structures}

The matrices $A$ ($T\times J$) and $F_{12t}$  are standard, dense covariate structures.  
The matrix $F_{11t}$ is given above, and is sparse.  Similary with $F_{2t}$.  
We separated out the time-invariant homogenous effects in $F_{12t}\Theta_{12}$, so 
the matrix $F_{12t}$ is $N\times K$.  These are non hierarchical and non time varying in their effects.   


\section*{Specifying parameters}

Table \ref{tab:parameters} summarizes the parameters that need to be
estimated, assuming that the covariance matrices are stationary.  The
number of parameters lists is the number of \emph{unique} elements.
For example, in a symmetric matrix there are, at most, $k(k+1)/2$
unique parameters.  But this can still be a large number, so we should
think about some kind of dimensionality reduction.
\begin{table}[h]\centering
  \begin{tabular}{lllll}
Symbol&Note&Num Pars (if dense)&Reduce by...&reduced parameters\\
\hline
$V_1$&symmetric pos-def&$N(N+1)/2$&make diagonal&$N$\\
&&&spatial structure&something $>N$\\
$V_2$&symmetric pos-def&\parbox{2.5cm}{$[N^2(1+P)^2+N(1+P)]/2$}&block diagonal?&$N(2+P(3+P))/2$\\
$W$&symmetric pos-def&$(1+J+P)(2+J+P)/2$&diagonal& (1+J+P)\\
&&&block diagonal& 1 + J(J+1)/2 + P(P+1)/2\\
$\delta$&scalar between 0 and 1&1&&\\
%$\lambda$ & scalar $>$zero\\\\
$\phi$ & dense matrix	&$J \times J$ &make symmetric & J(J+1)/2\\
$c,u$&\parbox{3cm}{vector, likely positive and close to zero, and may be negative}&J each&&\\
$\Theta_{12}$& \parbox{3cm}{time invariant homogenous coefficient matrix}&$K\times J$&no intercept&
  \end{tabular}
\caption{Parameters to be estimated}
\label{tab:parameters}
\end{table}

To estimate these parameters, we should transform them all to be
unbounded.  Otherwise, we need to modify the GDS algorithm to handle
constrained optimization and simulation (which is possible, but
tedious and uninteresting).

For the dense cases of $V_1$, $V_2$ and $W$, typically we would
estimate the elements of the lower Cholesky decomposition (taking logs
of the diagonal elements to ensure that they are positive).  If we add
structure to those matrices, we need to reconsider the
transformation.  However, block diagonals should still allow us to use
the Cholesky decomposition approach.  

\section*{Priors}

We need to choose initial values for $M_{20}$ and $C_{20}$.  What do
these matrices mean, intuitively?  They indicate prior information about the
starting states (the means across cities).  This could come from theory, another process, or could be made
quite diffuse.  For example, the way we specify the ad effectiveness (above) 
could provide us with some prior on the initial state that is constrained to be 
close to 0.  Similary, our understanding of price sensitivity is around $-1$ to $-2$
so we could provide such a information as a prior.   Of particular interest is
the "new" creative effectiveness.  The overall effect of this should be 
somewhat proportional to the effect of the base campaign.  

Prior on $\delta$:  if $\delta$ really is between 0 and 1, we could
make the prior uniform.  But is that realistic?  We probably want a
density that places zero probability at 0 and 1, or at least 0.  Does
the literature give us any prior information about what this decay
parameter should be?  Yes, it says that $\delta$ is usually around $0.1-0.2$ (for
weekly data).  We could use a beta$(1,3)$ as a prior.  One problem this
may raise (in the GDS) is that the parameter is constrained between zero and
one.  

Priors on $c_{1:J}$ and $u_{1:J}$:  depends on the domain.  Are they
all between 0 and 1.  Also, could they be correlated?  Would it make
sense that if one brand had high wearout, another brand could as well?
Do we have prior information on this?  Again, $c$ depends on the 
ad effectiveness value, but is unlikely to be much different from 0.  
The $w$ value depends on the scale used for advertising expenditure
and is expected to be positive.    The $\phi$ parameters are likely to be small
since it is unlikely any one creative can have a substantial impact on the 
overall effectiveness of the campaign.  

Priors on $V_1$, $V_2$ and $W$:  first, we need better intuition about
what these matrices represent.  Then, we can come up with a range of
reasonable values for the parameters.  Given the complexity of the
model, we will need to regularize it with prior information.  And it
would be good to give these priors careful though.  Way too many
marketers are careless with their ``uninformative'' priors.

\section*{Prediction}

Once we estimate these top-level parameters, we might want to simulate
data.  That means we need posterior predictive distributions of $Y$.
Can we do that without simulating the $\Theta$ parameters directly?
Note that we do not collect any $\Theta$ draws during the estimation
process, since they are all integrated out.  We should be able to use
forward filtering, backward sampling (possibly with smoothing) to obtain
the city level state variables.  


\subsection{Focus on initial qualities}
The accommodation of initial qualities could be a possible extension.   
Several possible outcomes could have some interesting implications.  First, 
consider if the distribution across creatives is quite broad, i.e. there is wide heterogeneity
in the quality of ads drawn (might want to discuss some of the creative agency literature). 
This would likely have a quite different outcome in terms of the decision to draw
new ads (and the number of creatives being displayed) than if the heterogeneity is
quite low.  As a "hunch" we may expect the manager to want to try more creatives, sticking
with the ones that seem to be high quality, but drawing new creatives if it is of low quality.
We may also expect that the pattern of expenditure is quite different - with more
pulsing behavior if the ad is high quality, but perhaps some initial pattern to 
'learn' about the quality of the advertisement once introduced.  Need to also consider
how pre-testing of creatives may change these results.  

The problem with the estimation of this is in the size of the evolution matrix required
which for just the advertising part (instead of a $J+1$ by $J+1$ matrix) requires a 
matrix which is of dimension equal to (with rather sloppy notation) $m+1$ by $m+1$ where $m$ is the total number of creatives across brands.   We start by seeing if we can 
collapse this across creatives somehow.  

\subsubsection{Additional to be done:}
The most important is to include interference effects for each advertising component.  
\begin{itemize}
\item Analytics on how to launch ads over time (constant noise).  Each ad has fixed
cost and variable cost of creating.  The variable cost is assumed measured by $g(\cdot)$.
\item How much does noise impact the pattern of spending for a focal brand?
\item Analytically derive the optimal number of creatives to run at any time, and
look at how this depends on the variability in inital quality.  Perhaps consider a 
choice of creative media agency?  One with high variability, but low average quality,
one with low variability?  If it's a beta distribution we could also look at the media agencies
that sometimes produce a hit, sometimes a complete flop, versus one that is more 
reliable "average" quality.  
\item Advanced - investigate optimal "pulsing" strategy
\end{itemize}


\end{document}



We go back to just one brand.  Define the "total impact" of any brand's advertising
as the part on the right hand side of (\ref{eqn:bqj1}):
\begin{equation}
\label{eqn:Baq}
\sum_{i=1}^m q_{ijt} I(\tau_{ji} \geq t) g(A_{ijt}) 
\end{equation}
Note that the indicator $I(\tau_{ji}\geq t)$ is associated with the quality parameter
rather than the quantity of the advertising.  This distinction is important - some readers
might note that having $g(A_{ijt})=0$ when the advertising for that creative has not started
would supplant the need to have this indicator. However, that
is not equivalent to the quality being given a zero weight.  Formally, 
$g(A_{ijt}) = 0$ does not imply that $I(\tau_{ji}\geq t)$ is zero, but $I(\tau_{ji}<t$ (the 
complement of $I(\tau_{ji}\geq t)$) does imply that $g(A_{ijt})=0$.  The indicator
is essential to keep track of which ads have started and which have not, and the
dynamics begin after $\tau_i$ for each ad.  If $g(A_{it})=0$ for any one period the
evolution of $q_{jit}$ continues based on the specification for wearout and forgetting.
 
We could rewrite the ad quality equation (\ref{eqn:Baq}) by multiplying by 
$\frac{\sum_{i=1}^m g(A_{ijt})}{\sum_{i=1}^m g(A_{ijt})}$.
So that we define the average quality (as a kind of geometric weighted average) as:
\begin{equation}
\tilde{q}_{jt}= \sum_{i=1}^m q_{ijt} I(\tau_{ji} \geq t) s(A_{ijt})  
\end{equation}
where $s(A_{ijt}) = \frac{g(A_{ijt})}{\sum_{i=1}^m g(A_{ijt})}$ and we can rewrite the brand equation in (\ref{eqn:bqj1}) as:
\begin{equation}
\label{eqn:bqj2}
B_{jt} = (1-\delta) B_{jt} + \tilde{q}_{jt} \sum_{i=1}^m g(A_{ijt})
\end{equation} 
With this specification, the question becomes how do we compactly write
an expression for the differential equation $\frac{d\tilde{q}_j}{dt}$?  We note that
this can be written as:
\begin{equation}
\label{eqn:dqt}
\frac{d\tilde{q}_j}{dt} = \sum_{i=1}^m \left[\frac{dq_{ij}}{dt} I(\tau_{ji} \geq t) s(A_{ijt}) + q_{ij} \frac{dI(\tau_{ji} \geq t) s(A_{ijt})}{dt} \right]
\end{equation}
We write the evolution of the quality of creative $i$ as:
\[
\frac{dq_i}{dt} = -a(A_{it})q_i + (1-I(A_{it}))\delta (q_{i\tau}^0-q_i) 
\] 
%So that we rewrite the average quality evolution in (\ref{eqn:dqt}) (recognizing for a normalized time increment $dt=1$, %$d\tilde{q}=\tilde{q}_t - \tilde{q}_{t-1}~~\forall t$) as:
%\begin{eqnarray}
%\tilde{q}_t & = & \tilde{q}_{t-1} + \sum_{i=1}^m  \left(-a(A_{it})q_{i,t-1} + (1-I(A_{it}))\delta (q_{i\tau}^0-q_{i,t-1})\right)  %I(\tau_{i} \geq t) s(A_{it})\nonumber\\
%& = & \sum_{i=1}^m q_{i,t-1} I(\tau_{i} \geq t-1) s(A_{i,t-1}) +  \sum_{i=1}^m  \left(-a(A_{it})q_{i,t-1} + (1-I(A_{it}))\delta %(q_{i\tau}^0-q_{i,t-1})\right)  I(\tau_{i} \geq t) s(A_{it})\nonumber
%\end{eqnarray}
 
Writing (\ref{eqn:bqj2}) as a function 
\begin{eqnarray}
\tilde{q}_t 	& = & \tilde{q}_{i,t-1} + \sum_{i=1}^m \left[dq_{it} I(\tau_i \geq t) s(A_{it}) + q_{i,t-1}  \left(I(\tau_i \geq t) s(A_{it})-I(\tau_i \geq t-1)s(A_{i,t-1})\right) \right]\nonumber\\
		& = & \sum_{i=1}^m q_{i,t-1} I(\tau_i \geq t-1) s(A_{it-1}) \nonumber\\
& & + \left[ \left(-a(A_{it})q_{i,t-1}+(1-I(A_{it}))\delta (q_{i\tau}^0-q_{it-1})\right)  s(A_{it}) + q_{i,t-1} \left( I(\tau_i \geq t) s(A_{it})- I(\tau_i \geq t-1) s(A_{i,t-1})\right) \right]\nonumber\\
& = & \sum_{i=1}^m q_{i,t-1} I(\tau_i \geq t-1) s(A_{it-1}) \nonumber\\
& & + \left( q_{i,t-1} \left(1-a(A_{it})-(1-I(A_{it}))\delta \right)+ (1-I(A_{it}))\delta q_{i\tau}^0\right)  I(\tau_i \geq t) s(A_{it}) \nonumber\\ &&+~q_{i,t-1}\left( I(\tau_i \geq t) s(A_{it})- I(\tau_i \geq t-1) s(A_{i,t-1}) \right) \nonumber\\
& = & \sum_{i=1}^m q_{i,t-1} I(\tau_i \geq t-1) s(A_{it-1}) \nonumber\\
& & + \left[\left( q_{i,t-1} \left(1-a(A_{it})-(1-I(A_{it}))\delta \right)\right) I(\tau_i \geq t) s(A_{it}) 
+ q_{i,t-1} \left(I(\tau_i \geq t) s(A_{it})-I(\tau_i \geq t-1) s(A_{i,t-1})\right) \right]\nonumber\\
&&+ (1-I(A_{it}))\delta q_{i\tau}^0I(\tau_i \geq t) s(A_{it})\nonumber\\
%%%%%
& = & \sum_{i=1}^m q_{i,t-1} I(\tau_i \geq t-1) s(A_{it-1}) \nonumber\\
& & + \left( q_{i,t-1} \left[\left(1-a(A_{it})-(1-I(A_{it}))\delta \right)\right) I(\tau_i \geq t) s(A_{it}) +  \left( I(\tau_i \geq t) s(A_{it})- I(\tau_i \geq t-1)s(A_{i,t-1})\right)\right]\nonumber\\
&&+ (1-I(A_{it}))\delta q_{i\tau}^0I(\tau_i \geq t) s(A_{it})\nonumber\\
& = & \sum_{i=1}^m q_{i,t-1} I(\tau_i \geq t-1) s(A_{it-1}) \nonumber\\
& &  q_{i,t-1} I(\tau_i \geq t-1) s(A_{it-1}) \left[\left(1-a(A_{it})-(1-I(A_{it}))\delta \right) \frac{I(\tau_i \geq t) s(A_{it})}{ I(\tau_i \geq t-1) s(A_{it-1})} +  \frac{I(\tau_i \geq t) s(A_{it})}{ I(\tau_i \geq t-1) s(A_{it-1})} -1 \right]\nonumber\\
&&+ (1-I(A_{it}))\delta q_{i\tau}^0I(\tau_i \geq t) s(A_{it})\nonumber\\
& = & \sum_{i=1}^m q_{i,t-1} I(\tau_i \geq t-1) s(A_{it-1}) \left(\left[\left(1-a(A_{it})-(1-I(A_{it}))\delta \right) \frac{I(\tau_i \geq t) s(A_{it})}{ I(\tau_i \geq t-1) s(A_{it-1})} +  \frac{I(\tau_i \geq t) s(A_{it})}{ I(\tau_i \geq t-1) s(A_{it-1})}  \right]\right)\nonumber\\
&&\sum_{i=1}^m (1-I(A_{it}))\delta q_{i\tau}^0I(\tau_i \geq t) s(A_{it})\nonumber\\
\end{eqnarray}

%\tilde{q}_{t}  & = & \sum_{i=1}^m q_{t-1} I(\tau_i \geq t-1) s(A_{i,t-1})+ \sum_{i=1}^m (q_{i,t}-q_{i,t-1}) I(\tau_{i} \geq t) %s(A_{i,t}) \nonumber\\ & + & q_{i,t} (I(\tau_{i} \geq t) s(A_{i,t})-I(\tau_{i} \geq t-1) s(A_{i,t-1}))\nonumber\\
%& = &  \sum_{i=1}^m q_{t-1} I(\tau_i \geq t-1) s(A_{i,t-1})+ \sum_{i=1}^m (q_{i,t}-q_{i,t-1}) I(\tau_{i} \geq t) s(A_{i,t}) %\nonumber\\ & + & q_{i,t} (I(\tau_{i} \geq t) s(A_{i,t})-I(\tau_{i} \geq t-1) s(A_{i,t-1}))\nonumber
