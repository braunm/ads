\documentclass[letter,11pt]{article}

\usepackage[pdftex]{graphicx}

\usepackage{geometry}
\usepackage{natbib}
\usepackage[singlespacing]{setspace}
\usepackage[page]{appendix}
\usepackage[color]{showkeys}

\definecolor{labelkey}{rgb}{1,.5,0.2}
\definecolor{refkey}{rgb}{0,0,0.8}

\usepackage[T1]{fontenc}
%\usepackage{float}
\usepackage{mathtools}
%%\usepackage{amsfonts}
\usepackage{amssymb}
\usepackage{verbatim}
\usepackage{mathpazo}
\usepackage{caption}

\bibpunct[, ]{(}{)}{;}{a}{}{,}
\bibliographystyle{ormsv080}

\DeclareMathOperator\logit{logit}
\DeclareMathOperator\Prob{Prob}
\DeclareMathOperator\vecop{vec }
\DeclareMathOperator\tr{tr}
\DeclareMathOperator\etr{etr}


\newcommand{\thetabar}{\bar\theta}
\newcommand{\thetaij}{\theta_{ij}}
\newcommand{\kron}{\otimes}
\newcommand{\eps}{\varepsilon}
\newcommand{\Igtz}{\mathbb{I}}
%%\newcommand{\Igtz}{\mathbf{I}}



\newcommand{\relphantom}[1]{\mathrel{\phantom{#1}}}

\geometry{left=1in,right=1in,top=1in,bottom=1in}

\title{Note on differential equation for HDLM}


\begin{document}
\maketitle

Let's start with the following differential equation:
\begin{align}
  \label{eq:de1}
  \frac{dq}{dt}&=-\left[c+uA\right]q+r(1-I(A))\left[b-q\right]
\end{align}

The symbols are defined as follows:

\begin{table}[h]\centering
\begin{tabular}[h]{cl}
  $q$&ad quality\\
$A$&ad spend\\
$c$&the rate at which quality attenuates towards zero in the absence of
   advertising\\
$u$&additional attenuation of quality that is proportional to ad
     spend\\
$\delta$&a decay parameter\\
$b$&a constant\\
$m$&long-run asymptote
\end{tabular}
\end{table}

If there is no ad spend, so $A=0$, then
$\lim_{t\to\infty}q_t=\dfrac{\delta}{c+\delta}b$.  If we define
$m=\dfrac{\delta}{c+\delta}b$, then the differential equation is
\begin{align}
  \label{eq:de2}
  \frac{dq}{dt}&=-\left[c+uA\right]q+\delta(1-\Igtz(A))\left[\frac{m(c+\delta)}{\delta}-q\right]\\
&=-\left[c+uA+\left(1-\Igtz(A)\right)\delta\right]q+(1-\Igtz(A))m(c+\delta)
\end{align}

Thus, if there is no ad spend, then $\lim_{t\to\infty}q_t=m$.

The discrete time version is:
\begin{align}
  \label{eq:de3}
  q_{t+1}=\left[1-c-uA-\left(1-\Igtz(A)\right)\delta\right]q_t+(1-I(A))m(c+\delta)
\end{align}

The evolution matrix is:

\begin{align}
  \label{eq:Gt}
  G_t=\begin{pmatrix}
(1-\delta)&\tilde{g}(A_{1t})&\cdots&\cdots&\tilde{g}(A_{Jt})\\
0&(1-c_1-u_1A_{1t}) +(1-\Igtz(A_{1t}))\delta&&&0\\
\vdots&&\ddots&&\vdots\\
\vdots&&&\ddots&0\\
0&\cdots&\cdots&0&(1-c_J-u_JA_{Jt}) +(1-\Igtz(A_{1t}))\delta
\end{pmatrix}
\end{align}

One component of the innovation matrix comes from the additive term in
the differential equation (check subscripts for $c_j$).

\begin{align}
  \label{eq:H1t}
  H_{1t}=\begin{pmatrix}
(1-I(A_{1t}))(c_1+\delta)m_{11}&\cdots&(1-I(A_{Jt}))(c_J+\delta)m_{1J}\\
\vdots&&\vdots\\
(1-I(A_{1t}))(c_1+\delta)m_{J1}&\cdots&(1-I(A_{Jt}))(c_J+\delta)m_{JJ}
\end{pmatrix}
\end{align}

If we add effects from creatives, then we add another component to the
innovation matrix.

\begin{align}
  \label{eq:H2t}
    H_{2t}&= \begin{pmatrix}
      E_{1t}&&&0\\
&E_{2t}\\
&&\ddots\\
0&&&E_{Jt}\\
  \end{pmatrix}
\begin{pmatrix}
\phi_{11}&\cdots&\phi_{1J}\\
\vdots&\vdots&\vdots\\
\phi_{J1}&\cdots&\phi_{JJ}
\end{pmatrix}
\end{align}

Here is how this all relates to the flags in the code.  If
\texttt{replenish} is \texttt{TRUE}, then we compute $H_{1t}$.
Otherwise, $H_{1t}=0$.    If
\texttt{include.phi} is \texttt{TRUE}, then we compute $H_{2t}$.
Otherwise, $H_{1t}=0$.  



\end{document}

