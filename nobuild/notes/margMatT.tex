\documentclass[letter,11pt]{article}

\usepackage[pdftex]{graphicx}

\usepackage{geometry}
\usepackage[singlespacing]{setspace}
\usepackage[page]{appendix}
%%\usepackage[color]{showkeys}

%%\definecolor{labelkey}{rgb}{1,.5,0.2}
%%\definecolor{refkey}{rgb}{0,0,0.8}

%\usepackage{float}
\usepackage{amsmath}
\usepackage{amssymb}
\usepackage{verbatim}
\usepackage{mathpazo}
\usepackage{caption}
\usepackage{parskip}

\geometry{left=1in,right=1in,top=1in,bottom=1in}
\DeclareMathOperator\sgn{sgn}
\DeclareMathOperator\Prob{Prob}
\DeclareMathOperator\diag{diag}
\DeclareMathOperator\erf{erf}



\newcommand{\thetabar}{\bar\theta}
\newcommand{\thetaij}{\theta_{ij}}
\newcommand{\kron}{\otimes}
\newcommand{\eps}{\varepsilon}
\newcommand{\Igtz}{\mathbb{I}}
\newcommand{\K}{\mathcal{K}}
\newcommand{\A}{\mathcal{A}}
\newcommand{\M}{\mathcal{M}}



\newcommand{\relphantom}[1]{\mathrel{\phantom{#1}}}


\usepackage[natbib=true,
citestyle=authoryear-comp,
style=authoryear,
bibencoding=utf8]{biblatex}

\ExecuteBibliographyOptions{maxcitenames=1,
  maxbibnames=99,
  citetracker=true,
  dashed=false,
  doi=false,
  mergedate=false,
  isbn=false}
  
\addbibresource{/Users/braunm/Documents/References/braun_refs.bib}


\setlength{\bibitemsep}{\parskip}
\AtEveryCitekey{\ifciteseen{}{\defcounter{maxnames}{3}}}
\DeclareFieldFormat[article,incollection,unpublished]{title}{#1} %No quotes for article titles
\DeclareFieldFormat[thesis]{title}{\mkbibemph{#1}} % Theses like book
                                % titles
\DeclareFieldFormat{pages}{#1} %% no pp prefix before page numbers
\renewbibmacro{in:}{%
  \ifentrytype{article}{}{ % Don't print In: for journal articles
  \printtext{\bibstring{in}\intitlepunct}} %% but use elsewhere
}
\renewbibmacro*{volume+number+eid}{%
  \printfield{volume}%
  \printfield{number}%
  \setunit{\addcomma\space}%
  \printfield{eid}}
\DeclareFieldFormat[article]{volume}{\mkbibbold{#1}}
\DeclareFieldFormat[article]{number}{\mkbibparens{#1}}
\DeclareFieldFormat[article]{date}{#1}
\AtEveryBibitem{\clearfield{day}}
\renewbibmacro*{issue+date}{% print month only
  \printtext{%
    \printfield{issue}%
    \setunit*{\addspace}%
\printtext{\printfield{month}}}
  \newunit}

\renewbibmacro*{publisher+location+date}{% no print year
  \printlist{location}%
  \iflistundef{publisher}
    {\setunit*{\addcomma\space}}
    {\setunit*{\addcolon\space}}%
  \printlist{publisher}%
  \setunit*{\addcomma\space}%
%%\printdate
  \newunit}

\renewcommand*{\nameyeardelim}{~} %no comma in cite






\title{Marginal Distribution of an Element of a Matrix-T R.V.}

\begin{document}
\maketitle
Let $X\sim MatT(\mu,R,Q,\nu)$, where $X$ and $\mu$ are $p\times q$
matrices, $R$ is a $p \times p$ matrix, and $Q$ is a $q \times q$
matrix.  The degrees of freedom is $\nu\ge p+q$. In the sequel, define
$c=\nu-p-q+1$.

Let $a$ be a $q$-dimensional vector and let $b$ be a  $p$-dimensional
vector.  Define
\begin{align}
  \label{eq:1}
  t=\frac{ca'(X-\mu)'b}{(a'Qa)(b'Rb)}
\end{align}

From \citet[p. 116]{KotzNadarajah2004}, $t$ has a Student's t
distribution with $c$ degrees of freedom.  Therefore,
\begin{align}
  \label{eq:2}
  f(t)=\frac{\Gamma(\frac{c+1}{2})}{\Gamma(\frac{c}{2})\sqrt{c\pi}}\left(1+\frac{t^2}{c}\right)^{-\frac{c+1}{2}}
\end{align}

Suppose $a_j$ is all zeros, except for a 1 in the $j^{th}$ element,
and suppose $b_i$ is all zeros, except for a 1 in the $i^{th}$
element.  That means that
\begin{align}
  \label{eq:3}
&a'(X-\mu)'b=x_{ij}-\mu_{ij}\\
&a'Qa=Q_{jj}\\
&b'Rb=R_{ii}\\
&t=\frac{c(x_{ij}-\mu_{ij})}{R_{ii}Q_{jj}} \\
&\frac{dt}{dx_{ij}}=\frac{c}{R_{ii}Q_{jj}}
\end{align}
The elements $R_{ii}$ and $Q_{jj}$ are diagonal elements from the
covariance matrices, so they are always positive.

Let
$\K=\displaystyle\frac{\Gamma(\frac{c+1}{2})\sqrt{c}}{\Gamma(\frac{c}{2})\sqrt{\pi}R_{ii}Q_{jj}}$.
By changing variables and including the Jacobian,
\begin{align}
  \label{eq:4}
  f(x_{ij})=\K\left(1+\frac{c(x_{ij}-\mu_{ij})^2}{R_{ii}^2Q_{jj}^2}\right)^{-\frac{c+1}{2}}
\end{align}

This is a non-standardized Student's t distribution with mean
$\mu_{ij}$ and scalaing parameter
$\sigma=\displaystyle\frac{R_{ii}Q_{jj}}{c}$.  Note that this is not
the same as a non-central t distribution, which is what is in R.  That
means we need to compute the pdf and cdf ourselves.

We want to know the probability that $x_{ij}<0$.  Let
$z=\displaystyle\frac{\mu\sqrt{c}}{R_{ii}Q_{jj}}$. Through the magic of
Mathematica, we get
\begin{align}
  \label{eq:5}
  F(0)=\frac{1}{2}-\K\mu~_2F_1\left(\frac{1}{2},\frac{c+1}{2};\frac{3}{2}; -z^2\right)
\end{align}

This is not so hard to compute, but we can simplify some more.  Applying the transformation in Equation 15.8.1 in \citet{NIST2010}.
\begin{align}
  \label{eq:6}
  \Prob(x_{ij}<0) =\frac{1}{2}-\K\mu \left(1+z^2\right)^{-\frac{1}{2}}~_2F_1\left(\frac{1}{2},1-\frac{c}{2};\frac{3}{2};\frac{z^2}{z^2+1}\right)
\end{align}

Using Equation 8.17.7 in \citet{NIST2010}, we can write this probability in
terms of an incomplete beta function $\mathbb{B}(\cdot)$.
\begin{align}
  \label{eq:7}
  \Prob(x_{ij}<0) &=\frac{1}{2}-\K\mu
  \left(1+z^2\right)^{-\frac{1}{2}}~_2F_1\left(\frac{1}{2},1-\frac{c}{2};\frac{3}{2};\frac{z^2}{z^2+1}\right)\\
&=\frac{1}{2}-\K\mu \left(1+z^2\right)^{-\frac{1}{2}}
\frac{1}{2}\left(\frac{z^2}{1+z^2}\right)^{-\frac{1}{2}}\mathbb{B}\left(\frac{z^2}{1+z^2};\frac{1}{2},\frac{c}{2}\right)
\end{align}

When we substitute back $\K$ and $z$, and remember that
$\sqrt{\pi}=\Gamma(\frac{1}{2})$, lots of stuff cancels out.  The
probability reduces to a \emph{regularized}
beta function $\widetilde{\mathbb{B}}(\cdot)$.
\begin{align}
  \label{eq:8}
  \Prob(x_{ij}<0)=\frac{1}{2}\left[1-\sgn(\mu)\widetilde{\mathbb{B}}\left(\frac{c\mu^2}{c\mu^2+R_{ii}^2Q_{jj}^2};\frac{1}{2},\frac{c}{2}\right)\right]
\end{align}

(The sgn function returns -1 if $\mu<0$, 0 if $\mu=0$ and 1 if $\mu> 0$).  Note that the regularized beta function is equivalent to the cdf of a
beta distribution.  This function is available in most statistical packages.

For automatic differentiation, we will need the derivative of the
incomplete beta function.
\begin{align}
  \label{eq:9}
  \frac{d\mathbb{B}(x;a,b)}{dx}=x^{a-1}(1-x)^{b-1}
\end{align}

Since $c$ is fixed and known, we do not need the derivative with
respect to the other arguments of the incomplete beta function.  This
makes life much easier.

It might be even faster to use a normal approximation to the
non-standard t distribution. As $c$ gets large, $f(x_{ij})$ approaches
a normal distribution with standard deviation
$\sigma=\displaystyle\frac{R_{ii}Q_{jj}}{c}$.  Since $c$ gets updated
after each period, the normal approximation might work very well, very
early.  Whether this is advisable depends on how fast the system can
compute the normal cdf.

\section{Specifics}

For updating $H_t$, we need to consider the posterior distribution of
$\theta_{2t}$:
\begin{align}
  \label{eq:10}
  \theta_{2t}\sim Mat\_T(M_{2,t-1},C_{2,t-1},\Omega_{t-1},\nu_{t-1})
\end{align}
The reason we use $t-1$ is that we are assuming advertising is
purchased at the beginning of the period, and the effectiveness of the
ads affects period $t$ sales.  Therefore, we cannot use period $t$
sales to update the latent parameters.  Besides, this is the only way
to get the recursion to work.

Bayesian updating of the inverse Wishart for $\Sigma$ gives us
\begin{align}
  \label{eq:11}
  \nu_t&=\nu_{t-1}+N\\
\Omega_t&=\Omega_{t-1}+(Y_t-f_t)'Q_t^{-1}(Y_t-f_t)
\end{align}
Because $\theta_{2t}$ has $1+J+P$ rows and $J$ columns, $\nu_0$ must
be greater than $P+2J+1$.  

Define
\begin{align}
  \label{eq:13}
S_t=\diag(C_{2t})\diag(\Omega_t)'
\end{align}
which is the outer product of the diagonals of
$C_{2t}$ and $\Omega_t$.

Any element $\theta{2,ijt}$ has a non-standardized Student t
distribution with parameters $M_{2,ijt}$,$S_{ijt}$ and $\nu_t-P-2J$.  Specifically,
\begin{align}
  \label{eq:14}
  f(\theta_{2,ijt})&=\frac{\Gamma\left(\frac{\nu_t-P-2J+1}{2}\right)}{\Gamma\left(\frac{\nu_t-P-2J}{2}\right)\sqrt{\pi
      (\nu_t-P-2J)} S_{ijt}}
\left(1+\frac{1}{\nu_t-P-2J}\left(\frac{\theta_{2,ijt}-M_{2,ijt}}{S_{ijt}}\right)^2\right)^{-\frac{\nu_t-P-2J+1}{2}}\\
\Prob(\theta_{2,ijt}<0)&=\frac{1}{2}\left[1-\sgn(M_{2.ijt})\widetilde{\mathbb{B}}\left(\frac{\left(\nu_t-P-2J\right)M_{2,ijt}^2}{\left(\nu_t-P-2J\right)M_{2,ijt}^2+S_{ijt}^2};\frac{1}{2},\frac{\nu_t-P-2J}{2}\right)\right]
\end{align}

When $\nu_t$ becomes large, we can use the normal approximation to the t.
\begin{align}
  \label{eq:15}
  \theta_{2,ijt}\sim N\left(M_{2,ijt},\frac{S_{ijt}}{\nu_t-P-2J}\right)
\end{align}

Writing the cdf at zero in terms of the error function,
\begin{align}
  \label{eq:16}
  \Prob(\theta_{2,ijt}<0)=\frac{1}{2}\left[1+\erf\left(\frac{-\left(\nu_t-P-2J\right)M_{2,ijt}}{\sqrt{2}S_{ijt}}\right)\right]
\end{align}





\printbibliography



\end{document}