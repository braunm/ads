\documentclass[letter,11pt]{article}

\usepackage[pdftex]{graphicx}

\usepackage{geometry}
\usepackage[singlespacing]{setspace}
\usepackage[page]{appendix}
%%\usepackage[color]{showkeys}

%%\definecolor{labelkey}{rgb}{1,.5,0.2}
%%\definecolor{refkey}{rgb}{0,0,0.8}

%\usepackage{float}
\usepackage{amsmath}
\usepackage{amssymb}
\usepackage{verbatim}
\usepackage{mathpazo}
\usepackage{caption}
\usepackage{parskip}

\geometry{left=1in,right=1in,top=1in,bottom=1in}
\DeclareMathOperator\sgn{sgn}
\DeclareMathOperator\Prob{Prob}
\DeclareMathOperator\diag{diag}
\DeclareMathOperator\erf{erf}
\DeclareMathOperator\sd{sd}
\DeclareMathOperator\logit{logit}
\DeclareMathOperator\sech{sech}



\newcommand{\thetabar}{\bar\theta}
\newcommand{\thetaij}{\theta_{ij}}
\newcommand{\kron}{\otimes}
\newcommand{\eps}{\varepsilon}
\newcommand{\Igtz}{\mathbb{I}}
\newcommand{\K}{\mathcal{K}}
\newcommand{\A}{\mathcal{A}}
\newcommand{\M}{\mathcal{M}}
\newcommand{\off}{\text{off}}



\newcommand{\relphantom}[1]{\mathrel{\phantom{#1}}}


\usepackage[natbib=true,
citestyle=authoryear-comp,
style=authoryear,
bibencoding=utf8]{biblatex}

\ExecuteBibliographyOptions{maxcitenames=1,
  maxbibnames=99,
  citetracker=true,
  dashed=false,
  doi=false,
  mergedate=false,
  isbn=false}
  
\addbibresource{/Users/braunm/Documents/References/braun_refs.bib}


\setlength{\bibitemsep}{\parskip}
\AtEveryCitekey{\ifciteseen{}{\defcounter{maxnames}{3}}}
\DeclareFieldFormat[article,incollection,unpublished]{title}{#1} %No quotes for article titles
\DeclareFieldFormat[thesis]{title}{\mkbibemph{#1}} % Theses like book
                                % titles
\DeclareFieldFormat{pages}{#1} %% no pp prefix before page numbers
\renewbibmacro{in:}{%
  \ifentrytype{article}{}{ % Don't print In: for journal articles
  \printtext{\bibstring{in}\intitlepunct}} %% but use elsewhere
}
\renewbibmacro*{volume+number+eid}{%
  \printfield{volume}%
  \printfield{number}%
  \setunit{\addcomma\space}%
  \printfield{eid}}
\DeclareFieldFormat[article]{volume}{\mkbibbold{#1}}
\DeclareFieldFormat[article]{number}{\mkbibparens{#1}}
\DeclareFieldFormat[article]{date}{#1}
\AtEveryBibitem{\clearfield{day}}
\renewbibmacro*{issue+date}{% print month only
  \printtext{%
    \printfield{issue}%
    \setunit*{\addspace}%
\printtext{\printfield{month}}}
  \newunit}

\renewbibmacro*{publisher+location+date}{% no print year
  \printlist{location}%
  \iflistundef{publisher}
    {\setunit*{\addcomma\space}}
    {\setunit*{\addcolon\space}}%
  \printlist{publisher}%
  \setunit*{\addcomma\space}%
%%\printdate
  \newunit}

\renewcommand*{\nameyeardelim}{~} %no comma in cite






\title{Transformations and priors that we use}

\begin{document}
\maketitle

In this document, we lay out three aspectsof model implementation:
\begin{enumerate}
\item Order that parameters are passed in to the estimation algorithm;
\item Transformations that we use to get from unconstrained to
  constrained parameters; and
\item Prior distributions of the \emph{unconstrained} parameters,
  including any Jacobians from the transformations.
\end{enumerate}

Parameters are passed in the following order:
\begin{enumerate}
\item $\theta_{12}$, columnwise, with no transformation.
\item $\bar{c}$,the average $c_j$, across $J$ brands (no transformation);
\item $\log\sd c$, the log of the standard deviation of $c_j$;
\item $c_{\off}$, the standardized offset of $c_j$ from the mean. Thus,
  $c_j=\sd(c)\left(c_{\off}-\bar{c}\right)$.
\item $\bar{u}$,the average $u_j$, across $J$ brands (no transformation);
\item $\log\sd u$, the log of the standard deviation of $u_j$;
\item $u_{\off}$, the standardized offset of $u_j$ from the mean.
  Thus,  $u_j=\sd(u)\left(u_{\off}-\bar{u}\right)$.
\item $\phi$, which is included only if the $H$ matrix is included.
  Columnwise, no transformation;
\item $\logit\delta$, a scalar parameter;
\item log diagonal elements for $V_1$ (see below for details);
\item factors for $V_1$, columnwise, some elements transformed (see
  details);
\item log diagonal elements for $V_2$ (see below for details);
\item factors for $V_2$, columnwise, some elements transformed (see
  details);
\item log scale parameter for $W_1$ (see below);
\item transformed Cholesky factors for $W_1$ (see below);
\item log diagonal elements for $W_1$ (see below for details);
\item factors for $W_1$, columnwise, some elements transformed (see
  details);
\end{enumerate}

\subsection{$\theta_{12}$}
The parameters for $\theta_{12}$ are passed in columnwise, with no
transformations. 

We apply a matrix normal prior, passing in a mean matrix, and the
lower Cholesky decompositions for the two covariance matrices.  The
matrix normal is a standard parameterization.

\subsection{$c$ and $u$}

The average effects $\bar{c}$ and $\bar{u}$ are unconstrained, with
normal priors.  The priors on $\sd{c}$ and $\sd{u}$ are
half-T. Hyperparameters are $\sigma_c$ and $\nu_c$ (similar for $u$). The
Jacobians of the transformations are $\sd{c}$ and $\sd{u}$.
Therefore,

\begin{align}
  \label{eq:2}
  \pi(\log\sd(c))&=\frac{2\Gamma(\frac{\nu_c+1}{2})}{\Gamma(\frac{\nu_c}{2})\sigma_c\sqrt{\nu_c\pi}}
\left[1+\frac{1}{\nu_c}\left(\frac{\sd(c)}{\sigma_c}\right)^2\right]^{-\frac{\nu_c+1}{2}}\sd(c)\\
  \pi(\log\sd(u))&=\frac{2\Gamma(\frac{\nu_u+1}{2})}{\Gamma(\frac{\nu_u}{2})\sigma_u\sqrt{\nu_u\pi}}
\left[1+\frac{1}{\nu_u}\left(\frac{\sd(u)}{\sigma_u}\right)^2\right]^{-\frac{\nu_u+1}{2}}\sd(u)
\end{align}

The prior for each $c_j$ is normal, with mean $\bar{c}$ and standard
deviation $\sd(c)$.  To operationalize this, we immediately transform
using the offsets
\begin{align}
  \label{eq:3}
  c_j&=\sd_c\left(c_{\off}+\bar{c}\right)\\
  u_j&=\sd_u\left(u_{\off}+\bar{u}\right)\\
\end{align}
This transformation lets us give each $c_{\off}$ and $u_{\off}$ a standard
normal prior (mean=0, sd=1).

\subsection{$\phi$}
The coefficient matrix $\phi$ is passed in columnwise, with no
transformation.  It is included only if we are including the $H$
matrix in the model.

We apply a matrix normal prior, passing in a mean matrix, and the
lower Cholesky decompositions for the two covariance matrices.  The
matrix normal is a standard parameterization.  

\subsection{$\delta$}
We pass in $\logit\delta$, and transform so
\begin{align}
  \label{eq:4}
  \delta&=\frac{\exp(\logit\delta)}{1+\exp(\logit\delta)}
\end{align}

The prior on $\delta$ is a beta distribution with parameters
$a_{\delta}$ and $b_{\delta}$.  The Jacobian of the
transformtion is $d\delta=\delta(1-\delta)$.  Therefore,
\begin{align}
  \label{eq:1}
  \pi(\logit\delta)&=\frac{\Gamma(a_{\delta}+b_{\delta})}{\Gamma(a_{\delta})\Gamma(b_{\delta})}\delta^{a_{\delta}}(1-\delta)^{b_{\delta}}
\end{align}
Note that when taking the log of this prior,
$\log\delta=\logit\delta-\log(1+\exp(\logit\delta))$, and
$\log(1-\delta)=-\log(1+\exp(\logit\delta))$.  This can be useful for
avoiding numerical issues that would result from computing unnecessary logarithms.

\subsection{$V_1$ and $V_2$}
The covariance matrices $V_1$ and $V_2$ are structured similarly, so
we will consider a general matrix $V$.  We will let $V$ take a
factor-analytic structure, where $x$ is a matrix of factors and
$\Sigma$ is a diagonal matrix with all positive elements. We construct
the matrix as
\begin{align}
  \label{eq:5}
  V=xx'+\Sigma
\end{align}

Let's start with $\Sigma$, and let the $i^{th}$ element of the
diagonal be $\Sigma_{ii}$.  The prior on each $\Sigma_{ii}$ is
half-T.  Since we are passing in $\log\Sigma_{ii}$ instead, we need to
multiply each half-T density by $\Sigma_{ii}$ (the Jacobian of the
transformation).

We arrange $x$ to have each column be a factor.  For identification
(need a reference for this), the upper triangle of $x$ is zero, and
the elements of the diagonal are all positive.  Therefore, if $x$ has
$k$ rows and $n$ columns, there are only $kn-\frac{1}{2}n(n+1)$
unconstrained parameters and $\frac{1}{2}n(n-1)$ positive parameters.
The unconstrained parameters all have T priors, and the
positive ones have half-T priors.  Only the half-T densities need to
be multiplied by the Jacobian of the transformation.

When passing in the parameters, $\log\diag\Sigma$ comes first,
followed by the elements of $x$, columnwise.

If there are no factors, then $V=\Sigma$.

\subsection{W}

Now, it gets fun.  We partition $W$ so the upper left corner is
$\alpha W_1$
and the lower right corner is $W_2$.  The upper right and lower left
corners are all zero.  $W_2$ has the same factor-analytic structure as
$V_1$ and $V_2$.

For identification, all elements of the diagonal of $W_1$ must be the
same.  Therefore, we treat $\alpha W_1$ as a scaled correlation matrix,
where $\alpha>0$ is the scale parameter is $W_1$ is a symmetric, positive definite
matrix will all ones on the diagonal.  The prior on $\log\alpha$ is a
half-T prior, multiplied by $\alpha$ (the Jacobian).

We place an LKJ prior on $W_1$, with parameter $\eta$ (see LKJ paper
and Stan manual).  $W_1$ has $J+1$ rows/columns.
\begin{align}
  \label{eq:6}
\pi(W_1)=C|W_1|^{\eta-1}
\end{align} where
\begin{align}
  C&=2^{\sum_{i=1}^J(2\eta-2+J+1-i)(J+1-i)}
\prod_{i=1}^J\left[\mathbb{B}\left(\eta+\frac{J-i}{2},\eta+\frac{J-i}{2}\right)\right]^{J+1-i}
\end{align}

Let $y$ be the vector of the $d=\binom{J+1}{2}$ unconstrained
parameters, arranged columnwise in a lower triangular matrix.  Let
$z_{ij}=\tanh(y_{ij})$ (this is a Fisher transformation).  Following
the procedure in the Stan manual, construct another lower triangular
matrix $x$ as follows:
\begin{enumerate}
\item $x_{11}=1$
\item For $i=2\mathellipsis J+1$, $x_{i1}=z_{i1}$ (copy first column)
\item For $i=2\mathellipsis J+1$,
  $x_{ii}=\prod_{k=1}^{i-1}\sqrt{1-z_{ik}^2}$ (diagonal elements)
\item For $j=2\mathellipsis J+1$ and $i=j+1\mathellipsis J+1$,
  $x_{ij}=z_{ij}\prod_{k=1}^{j-1}\sqrt{1-z_{ik}^2}$ (remaining
  off-diagonal elements)
\end{enumerate}
This transformation will ensure that $W_1=xx'$ is a valid correlation
matrix.

Through the magic of Mathematica, we get the following Jacobian of the transformation
\begin{align}
  \label{eq:8}
  \mathcal{J}_{W_1}&=\prod_{j=1}^{d-1}\prod_{i=j+1}^d\left[\sech(y_{ij})\right]^{d-j+1}\\
&=\prod_{j=1}^{d-1}\prod_{i=j+1}^d\left[1-z_{ij}^2\right]^{\frac{d-j+1}{2}}
\end{align}
The operator $\sech$ is the hyperbolic secant.  The second line comes
from the identity $\sech^2(x)=1-\tanh^2(x)$.

Mathematica also tells us that
\begin{align}
  \label{eq:9}
  |W_1|&=\prod_{j=1}^{d-1}\prod_{i=j+1}^d\left[1-z_{ij}^2\right]
\end{align}


If $\eta=1$, the distribution is uniform over all correlation
matrices.  For $\eta>1$, there is a mode at the identity matrix, and
for $\eta<1$ there is a trough.




\printbibliography



\end{document}